\chapter{Introduction}
The neutrino was originally proposed by Pauli in 1930 \cite{pauli} to conserve energy and momentum in the beta decay process,
\begin{equation}
n \rightarrow p + e^- + \bar{\nu}_e,
\end{equation}
for which the electron was earlier shown to have a continuous energy spectrum \cite{chadwick}.
Being a low mass, neutral particle, it was some 25 years before the first detection of neutrino interactions in the Cowan-Reines experiment \cite{cowan-reines} through inverse beta decay,
\begin{equation}
\bar{\nu}_e + p \rightarrow n + e^+.
\end{equation}
Later, the fact that neutrinos come in three flavors was established with the detection of muon neutrinos \cite{danby} and tau neutrinos \cite{donut}, with the earlier discovered particles identified as electron (anti-) neutrinos.

As neutrinos were known to be produced in nuclear reactions, the Homestake experiment \cite{homestake} was proposed in the 1960s to measure the flux of neutrinos produced in the core of the Sun.
This experiment measured significantly lower flux than was expected from theoretical calculations, becoming the so-called "solar neutrino problem" (SNP).
Many experiments \cite{sage,gallex,gno} confirmed this deficit, and it persisted until the turn of the century when the SNO experiment \cite{3phase} made a measurement of the solar neutrino flux including all neutrino flavors finding that the initially electron-flavor solar neutrinos were transformed into muon- or tau-flavor neutrinos before arriving at Earth.
Flavor change in the neutrino sector was confirmed by the Super-Kamiokande \cite{superk} experiment by observing an up/down anisotropy in atmospheric muon neutrinos that would not be expected if muon neutrinos remained muon flavor.
This discovery of flavor change implied a nonzero neutrino mass, and a set of massive neutrino states.
Superpositions of the mass states form the flavor states described earlier.

Determining the parameters of the mixing matrix that relates the flavor and mass basis has been the forefront of neutrino research for the last 20 years, and has seen great success measuring the rotation between the two bases and the magnitude of mass difference between mass states. 
Currently, the absolute masses of the neutrino states have yet to be determined, with the best constraints coming from cosmology, which still only sets a limit on the sum of the mass states \cite{pdg}.
Further, the origin of neutrino mass, be it Dirac or Majorana, is an open question being explored by the current generation of detectors, such as SNO+ \cite{snop}, Cuore \cite{cuore}, and LEGEND \cite{legend}, by studying a process called neutrinoless double beta decay (beta decay where the neutrinos, as their own antiparticle annihilate, and do not appear in the final state).


In this thesis I describe two analyses of solar neutrino data from the Sudbury Neutrino Observatory (SNO) and its upgrade, SNO+.
The first analysis, described in \Cref{ch:lifetime}, explores the possibility that neutrinos are unstable particles and can decay into more stable states.
Using solar neutrino data from SNO and combining with other solar neutrino experiments, a limit is set on the rate of such decay. 
The second analysis in \Cref{ch:es} uses commissioning data from SNO+ to measure the flux of $^8$B solar neutrinos. 
To aid in understanding these topics, the remainder of this chapter is devoted to the theory of neutrinos and particulars of solar neutrinos, while \Cref{ch:detectors} discusses the steps necessary to detect and analyze solar neutrino fluxes.
Paired with these analyses are, in \Cref{ch:chsrc}, the design and evaluation of an optical calibration source intended for later phases of SNO+, and, in \Cref{ch:wbls} an evaluation of the optical properties of a next-generation target material for the T\textsc{HEIA} detector.
But first, the theory.

%\section{Standard Model Neutrinos}
%[... maybe ...]

\section{Neutrino Oscillation}
\label{ch:theory}

Neutrinos are produced and interact in the flavor basis, $\ket{\nu_{\alpha}}$ where $\alpha = {e,\mu,\tau}$, however these are not eigenstates of the vacuum Hamiltonian, whose eigenstates (the eigenstates with definite mass) are denoted as $\ket{\nu_i}$ where $i = {1,2,3}$. 
The flavor basis is related to the mass basis by the Pontecorvo-Maki-Nakagawa-Sakata (PMNS) matrix $U_{\alpha i}$ as follows:
\begin{equation}
\ket{\nu_{\alpha}} = \sum_i U^*_{\alpha i} \ket{\nu_i}.
\end{equation}
The free evolution of these states is most easily represented in the mass basis, as these are eigenstates of the Hamiltonian:
\begin{equation}
\bra{\nu_i}H_{0}\ket{\nu_j} = \frac{1}{2E}\begin{bmatrix}
m_1^2 & 0 & 0 \\
0 & m_2^2 & 0 \\
0 & 0 & m_3^2
\end{bmatrix}.
\end{equation}
The survival probability for flavor state $\ket{\nu_\alpha}$ to be detected as flavor state $\ket{\nu_\beta}$ at some later time after free evolution for a distance L is therefore
\begin{equation}
P_{\alpha\beta} = \left|\braket{\nu_\beta(t)}{\nu_\alpha}\right|^2 = \left|\sum_i U^*_{\alpha i} U^{}_{\beta i} e^{- i m_i^2 L/(2E)} \right|^2.
\end{equation}

\section{Solar Neutrinos}

The Sun is powered by thermonuclear reactions that fuse light nuclei into heavier nuclei, releasing energy.
Much of this energy is in the form of electromagnetic radiation or energetic charged particles which thermalizes and is radiated away long after the initial reaction.
However, energetic neutrinos produced in this reactions escape quickly and relatively unhindered, making solar neutrinos an excellent probe of the internal dynamics of the Sun.
The following sections discuss the specific processes resulting in solar neutrinos, how the fluxes of solar neutrinos are predicted, and what other considerations must be taken into account to understand the solar neutrino fluxes at Earth.

\subsection{Fusion Reactions}

The primary reaction in the Sun is the proton-proton ($pp$) reaction,
\begin{equation}
p+p \rightarrow ^2\mathrm{H}+e^++\nu_e,
\end{equation}
which produces the most neutrinos, but has the lowest energy, making $pp$ neutrinos quite hard to detect.
The $^2$He from this reaction is involved in further reactions in the proton-proton chain, resulting in other neutrino fluxes.
Notable among these is the $^8$B reaction which produces neutrinos that are both high energy and relatively high flux, making them an ideal candidate for detection in water Cherenkov detectors.
Beyond the proton-proton chain, there is also the so called CNO process, involving carbon, oxygen, and nitrogen nuclei. 
Neutrinos produced from the CNO process have energies and fluxes between $pp$ and $^8$B neutrinos.
\Cref{neutrino_spectra} shows the energy spectra and total flux of all known solar neutrinos, where the fluxes represent theoretical predictions.

\begin{figure}
\centering
Here be Bachall's figure.
\caption{\label{neutrino_spectra}Shown here are the fluxes and spectra of solar neutrinos as calculated in \cite{bs05op}.}
\end{figure}

\subsection{Standard Solar Models}

The fluxes shown in \Cref{neutrino_spectra} are from a theoretical calculation of the equilibrium state of the Sun known as a Standard Solar Model (SSM).
These calculations use hydrodynamic equations of state, contemporary solar observations, and estimations of the primordial composition of the solar system to infer the properties of the solar core.
These properties, such as temperature, density, and isotope content, directly control the rates of the various fusion reactions, and therefore predict the fluxes of solar neutrinos.
These rates depend strongly on predictions for nuclear cross sections at densities and temperatures consistent with the solar core, and uncertainties here drive the uncertainties in the neutrino fluxes.

\subsection{The MSW Effect}

The MSW~\cite{wolfenstein,mikheyev} effect proposes that the coherent forward scattering of electron flavor neutrinos off of electrons in a material adds a potential energy, $V_e$, to electron flavor neutrinos which depends on the local electron density, $n_e$:
\begin{equation}
V_e = \sqrt{2} G_F n_e.
\end{equation}
Written in the mass basis the Hamiltonian including this effect, $H_{MSW}$, then takes the form
\begin{equation}
H_{MSW} = H_{0} + U\begin{bmatrix}
V_e & 0 & 0 \\
0 & 0 & 0 \\
0 & 0 & 0
\end{bmatrix}U^\dagger.
\end{equation}
Notably the evolution of the states in the presence of matter is now much more complicated since the eigenstates now depend on the electron density.

It is useful to introduce the matter mass basis, $\ket{\nu_{mi}(V_e)}$, consisting of eigenstates of the Hamiltonian $H_{MSW}$ at a particular electron potential $V_e$.
Note that if the electron density is zero, $H_{MSW}$ reduces to $H_0$. Therefore $\ket{\nu_{mi}(0)} \rightarrow \ket{\nu_{i}}$.
In many cases the variation of $V_e$ is slow enough that the evolution is adiabatic, meaning some state $\ket{\nu(t)}$ has a constant probability to be one of the instantaneous matter mass states at a later time. 
\begin{equation}
\left| \braket{\nu_{mi}(V_e(0))}{\nu(0)} \right|^2 = \left| \braket{\nu_{mi}(V_e(t))}{\nu(t)} \right |^2
\end{equation}

The electron density within the sun varies sufficiently slowly for an adiabatic approximation to be made.
Therefore, knowing where in the Sun a neutrino is produced (or more precisely the electron density at the production point), one can calculate the eigenstate composition for as long as the adiabatic condition is satisfied.
Once the neutrino reaches the solar radius, vacuum propagation dominates. 
As vacuum propagation does not change the mass state composition of a state, the neutrinos that arrive at Earth have the same mass state composition as those exiting the Sun.
Due to the large distance between the Earth and the Sun, these mass state fluxes can be assumed to be incoherent once they arrive at Earth, and any regeneration of coherence in the Earth is ignored.
Therefore, the arrival probability $\phi_i$ of neutrino mass state $\nu_i$ at Earth due to electron neutrinos produced at an electron potential $V_e$ in the Sun in the presence of the MSW effect can be calculated as
\begin{equation}
\phi_i = \left| \braket{\nu_{m i}(V_e)}{\nu_e} \right|^2.
\label{rawflux}
\end{equation}
The analytic expression for this value is non-trivial and in practice $H_{MSW}$ is numerically diagonalized in the flavor basis to find $\bra{\nu_{m i}(V_e)}$ at a particular $V_e$ value and compute this projection.

Ultimately the MSW effect is the leading order effect responsible for the observed survival probabilities of solar neutrinos.
\Cref{fig:global_solar} shows all the solar neutrino flux measurements made to date overlaying the expected MSW survival probability.
With the leading order effects now understood, the door is open for precision measurements of neutrino, and particular solar neutrino, interactions.

\begin{figure}
\centering
A figure goes here
\caption{\label{fig:global_solar}Shown here are the measured fluxs of solar neutrinos at various energies by many solar experiments to date.}
\end{figure}

\chapter{Solar Neutrino Detectors}
\label{ch:detectors}

Neutrinos only interact weakly with normal matter, so the interaction cross sections are quite small, being less than $10^{-40}$ cm$^2$ for energies relevant to solar neutrinos.
Even with expected fluxes of detectable solar neutrinos exceeding $10^6$ cm$^{-2}$s$^{-1}$, the interaction rates are minuscule.
Because of this, neutrino detectors typically rely on large masses of target material to achieve a reasonable interaction rate, with a solar neutrino event a day being quite respectable.
These low rates mean that neutrino detectors must be isolated from cosmic radiation by being constructed deep underground, or lose the neutrino events in the noise.
Even then the materials used to construct the detector must have very low intrinsic radioactivity or suffer the same fate.

The earliest neutrino detectors relied on the transmutation of nuclei by neutrino interactions. 
In the Homestake \cite{homestake} experiment, chlorine nuclei were transformed to argon nuclei via a neutrino capture,
\begin{equation}
\nu_e + ^{37}\mathrm{Cl} \rightarrow ^{37}\mathrm{Ar} + e^-,
\end{equation}
and the created argon was periodically collected its decays counted to estimate the interaction rate in the target volume.
The gallium experiments GNO \cite{gno}, GALLEX \cite{gallex}, and SAGE \cite{sagecombo}, relied on a similar method with gallium:
\begin{equation}
\nu_e + ^{71}\mathrm{Ga} \rightarrow ^{71}\mathrm{Ge} + e^-.
\end{equation}
Notably detectors relying on these radiochemical methods were only sensitive to $\nu_e$ and the solar neutrinos that had oscillated into another flavor were not detected, resulting in the SNP.
Also of note is that these detectors provide no information about the neutrinos that interact besides their rate: energy and direction information is totally lost in the radiochemical detection scheme.

The next generation of neutrino detectors progressed to real time detection of neutrino interactions.
These were the water Cherenkov detectors, such as SNO \cite{3phase} and Super-Kamiokande \cite{superk}, which instrumented a large volume of water with sensitive light detectors that captured flashes of Cherenkov light from energetic particles produced in the neturino interaction.
Pure water detectors like Super-Kamiokande are dominated by an elastic scatter interaction,
\begin{equation}
\nu_e + e^- \rightarrow \nu_e + e^-,
\end{equation}
where the final state electron has a momentum highly correlated with the incident neutrino.
Analysis of data collected by such detectors allows for a statistical determination of the properties of the neutrino flux that produced such interactions.
Real time detection also allows for different interaction channels to be distinguished, a feature leveraged by the SNO detector, which was instrumental in solving the SNP.
From water Cherenkov detectors, the field has progressed to target media other than water, such as the use of liquid scintillators in Borexino \cite{borexino}, which can provide complementary information and lower energy thresholds compared to water. 

In the following sections I will focus on SNO, SNO+, and Theia as they are directly applicable to analyses presented later.
As mentioned, the SNO detector was a water Cherenkov detector, but using heavy water. 
Its upgrade, SNO+, aims to replace the water target with a pure liquid scintillator.
Finally, the Theia detector is a proposed next-generation neutrino detector using a target medium that combines the best features of water and liquid scintillator.

\section{SNO}

The SNO \cite{sno} detector is located deep underground near Sudbury, ON, CA in the Creighton mine, an active nickle mine.
This location is one of the deepest laboratories in the world at $2$ km underground providing a 5800 m.w.e. overburden to shield from cosmic rays.
The detector itself is composed of a 6 m radius acrylic vessel (AV) in the shape of a spherical shell containing 1 kT of heavy water ($^2$H$_2$O or D$_2$O) as the active volume.
The AV was accessible by a large tube through which calibration sources could be deployed.

Surrounding the AV is a 9 m radius steel geodesic structure containing 9500 inward-facing photomultiplier tubes (PMTs) called the PMT support structure (PSUP).
These PMTs are sensitive to single photons and are intended to capture the Cherenkov radiation from energetic charged particles within the AV.
The entire structure is submerged in a large barrel shaped cavity of light water (H$_2$O) to shield the heavy water in the AV from the natural radioactivity of the mine and other detector materials.
Both the heavy water inside the AV and the light water of the cavity were continually purified to keep intrinsic radioactivity low and cooled to minimize biological growth.

The PMTs used in SNO were 8-inch Hamamatsu R1408 created with low-radioactivity glass. 
These PMTs had a time resolution for detected photons of 1.5 ns, and a peak quantum efficiency of 21.5\% at 440 nm.
Each PMT was outfitted with non-imaging light-concentrating reflectors to increase the effective coverage of light-sensitive elements to 55\%.
To assist in vetoing cosmic rays, 91 additional PMTs were mounted to the PSUP looking outward into the light water of the cavity. 

Whenever a PMT detected a photon, a discriminator would fire on the corresponding voltage fluctuation and register a hit.
This hit started a voltage ramp on a time-to-amplitude converter (TAC) to track the hit time, and integrated the total charge collected by the PMT.
Additionally, the PMT readout electronics would emit two $10mV$ square pulses of configurable widths which are summed across the entire detector and act as a count of coincidentally hit PMTs within each window.
For SNO these windows were set to 20ns and 100ns.
The trigger system discriminated the analog sum of each trigger, and, when it crossed a configurable threshold, the trigger system would issue a global trigger (GT) causing all PMTs hit in the last 400ns to report the value of their TAC (representing the time before the global trigger) and integrated charge (representing the number of detected photons - typically one).
For SNO the discriminator threshold on trigger sums corresponded roughly to 17 hit PMTs.
The hits from each GT are built into a single event for further analysis.

\subsection{Interactions}

The deuterium in the heavy water target used in SNO allows for multiple types of interactions sensitive to different combinations of neutrino flavors.
Most important is the neutral current interaction NC due to a neutrino scattering off a quark as shown in \Cref{fig:NC}.
NC is equally sensitive to all flavor of neutrino, allowing for an oscillation-agnostic measure of the total neutrino flux.
In practice the energy deposited in the quark results in a detectable signal if it exceeds the binding energy of the proton and neutron forming a deuteron.
The liberated neutron will eventually capture on some nuclei, resulting in a cascade of gammas which can scatter electrons with sufficient energy to produce Cherenkov light.
Due to the detection scheme, much of the information of the initial neutrino is lost in NC interactions.

\begin{figure}
\centering
\caption{\label{fig:NC}The Feynman diagram for neutral current (NC) interactions with quarks.}
\end{figure}

A charged current (CC) interaction with quarks in a deuteron as in \Cref{fig:CC} can convert the neutron into a proton and break up the deuteron with sufficient energy.
This interaction converts the neutrino into an energetic electron with an energy correlated with the incident neutrino.
In principle any neutrino flavor could participate in this interaction with sufficient energy to create heavier leptons, however solar neutrinos only have enough energy for electron flavor neutrinos to participate here.
The energetic electron then produces Cherenkov light to be detected.
This interaction channel allows for the recovery of spectral information due to the correlation of the electron energy to the neutrino.

\begin{figure}
\centering
\caption{\label{fig:CC}The Feynman diagram for charged current (CC) interactions with quarks.}
\end{figure}

Finally, there is the elastic scatter (ES) of electrons present in other water Cherenkov detectors and shown in \Cref{fig:ES}.
For electron neutrinos, both W and Z bosons can be exchanged.
However, like the CC interaction, solar neutrinos do not have enough energy to produce heavier leptons, so only Z exchange is possible.
This means the ES interaction is sensitive to all flavors of neutrinos, however non-electron flavor neutrinos have a lower cross section.
Due to kinematics, the final state electron energy is correlated with the incident neutrino energy.
Additionally, the direction (momentum) of ES electrons is highly correlated with the incident neutrino direction.
This means that ES interactions give both spectral and pointing information, allowing for neutrino sources to be distinguished not only by their energy signature, but based on their origin in space.
For solar neutrinos, this means ES electrons will point away from the Sun.

\begin{figure}
\centering
\caption{\label{fig:ES}The Feynman diagrams for elastic scatter (ES) interactions with electrons.}
\end{figure}


\subsection{Experimental Phases}

SNO operated in three distinct phases differing in their neutron detection capabilities, and hence their sensitivity to the NC interaction.
These phases are described in the following sections along with the neutron detection mode in each phase.

\subsubsection{D$_2$O}

The first phase consisted of pure D$_2$O. 
Neutrons produced in the NC interaction primarily captured on other deuterons in the target.
The produced $^3$H then decayed, releasing a 6.26 MeV gamma.
This gamma could either scatter electrons in the target or create energetic electron-positron pairs, either mode being detected by Cherenkov light.

\subsubsection{Salt}

In the second phase, NaCl was added to the D$_2$O.
Chlorine has a much greater neutron capture cross section than deuterium, resulting in a significant increase in neutron detection efficiency.
The decay scheme after a neutron capture on Cl is quite complicate, releasing many gammas, and giving these events a distinct topology compared to other interactions.
Neutron detection was further enhanced by an increase in total energy released in the gamma cascade to 8.6 MeV.

\subsubsection{NCD}

Finally, in the third phase, the NaCl was removed and neutron counting devices (NCDs) were added inside the AV.
These were essentially independent detectors, allowing for an independent measure of the neutron production rate.
A single NCD was a high purity nickel tube containing $^3$He gas, and was instrumented to utilize the $^3$He as a proportional counter for thermal neutrons~\cite{ncd}.
In total, 40 NCDs were added, two of which were modified to contain $^4$He (insensitive to neutrons) to understand instrumental backgrounds.
The NCD data is not directly used in the analysis described in the next chapter, however it was independently analyzed to constrain the analysis of the PMT data \cite{sno_ncd_psa,3phase}.

\subsection{Cherenkov Radiation}

Ultimately all light detected in SNO was the result of electrons being scattered with sufficient energy to exceed the local speed of light, commonly called Cherenkov radiation \cite{cherenkov}.
This light is emitted at an angle dependent on the index of refraction of the material and the momentum (direction) of the particle. 
Because of this, the geometry of the detected hits from Cherenkov radiation can be used to infer the average direction of the electron with resolution of order 10 degrees.
In the cases of ES or CC events, where the electron direction is correlated to the primary neutrino direction, this information can be used to disentangle solar neutrino events from backgrounds.
This was leveraged in analyses of SNO data and is a very benificial feature when dealing with directional sources.

\subsection{Vertex Reconstruction}

To perform an analysis of SNO data it is first necessary to extract physical observables from the PMT hit time information provided by the DAQ.
This is done using pattern recognition algorithms which take into account the geometry of cherenkov light to reconstruct the direction, position, and time of each event.
Using that information, along with the total number of hits, other algorithms can estimate the energy deposited in the detector. 
In SNO there were roughly 10 detected hits per 1 MeV of deposited energy.
Further quantities can be extracted, such as a measure of the isotropy of the event, $\beta_{14}$, and the ratio of hits within a prompt time window to all hits, ITR, encode additional information about the event that can, for instance, be useful for selecting physics events from instrumental events.

\subsection{Simulation and Analysis}

The physical observables determined from vertex reconstruction can then used to statistically sort events into different classes of interactions.
This statistical sorting is done with a maximum likelihood fit where the number of events in each class of interaction is floated, and each class of event is described by a probability distribution function (PDF) of some observable quantities.
In SNO the observable quantities of choice are:
\begin{enumerate}
    \item Energy - the estimated deposited energy assuming Cherenkov light from an electron.
    \item Radius - the volume weighted radius of the event estimated from reconstruction.
    \item Isotropy - the $\beta_{14}$ parameter based on Legendra polynomials giving a direction independent measure of isotropy of hits.
    \item Direction - the cosine of the angle between the event direction and the direction to the Sun at the time of the event.
\end{enumerate}
These observables were chosen to maximize separation between different types of physics interactions (NC, CC, ES) and of physics, backgrounds, and instrumentals. 
To properly take into account correlations between these parameters, 4D PDFs were used instead of four separate 1D PDFs for each event class.

The likelihood fit determines the realitive contribution of each event class necessary to best match the observed data.
To do this, the fit needs the 4D PDF for each type of event, and these PDFs were generated using a Monte Carlo method.
SNOMAN \cite{sno_nim} was a fortran simulation package developed by SNO that was adjusted to reproduce calibration data taken with the detector.
The reconstruction algorithms used on raw data were applied to the simulation results of SNOMAN for known classes of interactions to produce the PDFs used in analysis.
Because there may be systematic differences between simulated data and real data, even after calibration, systematic uncertainties on the scale and offset of each axis were considered.
For some types of events, such as the radioactivity of the PMT glass, it was impractical to simulate a sufficient number of interactions to produce reliable PDFs.
In these cases, analytical forms were used instead, and the parameters controling these analytical forms were estimated from the data.

\subsubsection{Solar Event Classes}

Solar neutrino events (ES, CC, NC) were generated for particular neutrino energies, however neutrino energy is not an observable of the data.
To extract information that depends on neutrino energy, such as the electron neutrino survival probability $P_{ee}$, one can use a model that defines the relative contribution of different neutrino energies to the PDF for that class of event.
For instance, since all CC events must be electron type neutrinos, the CC PDF events would be weighted by $P_{ee}$ in addition to the expected energy spectrum. 
If the parameters for $P_{ee}$ are floated in the likelihood fit, one can simultaneously estimate the model parameters from the data.
This will be discussed further in \Cref{ch:lifetime} focusing on the specifics of the neutrino lifetime analysis.

\section{SNO+}

The SNO experiment finished taking data in the early 2000s, and an upgrade of the detector was planned which ultimately led to SNO+ \cite{sno+}.
The primary goal of SNO+ is to perform a neutrinoless double beta decay measurement (0$\nu\beta\beta$) using the isotope $Te_{130}$, however that is beyond the scope of this thesis.
What is relevant is the change in target media from SNO to SNO+: heavy water will be substituted for a light water calibration phase and a scintillator data taking phase.

Water phase in SNO+ is primarily intended to calibrate the detector and test the upgraded hardware in a regime similar to SNO.
Light water differs from heavy water primarily in the available interaction channels: neither NC nor CC are possible given the lack of targets (deuterons) with sufficiently low interaction thresholds.
The ES interaction is unaffected by this change in targets, since both have roughly the same electron density, and \Cref{ch:es} goes into detail on measuring the $^8$B solar neutrino flux with this channel.
Additionally, as part of the calibration effort in water phase, \Cref{ch:chsrc} describes an optical calibration source designed for SNO+.
Data taking for this phase began mid 2017 and finished toward the end of 2018.

The change to pure scintillator LAB+PPO (linear alkylbenzene with a primary flour 2,5-diphenyloxazole at 2 g/L) started at the end of 2018.
This switch was motivated by a need for enhanced energy resolution for the 0$\nu\beta\beta$ measurement, and similar reasoning comes into play in its applicability to solar neutrinos.
In terms of solar neutrinos, the ES interaction channel is still the only relevant one, however the observed signals in scintillator phase will be quite different than that of water phase.
In water all information about the interaction was gathered from the Cherenkov light emitted. 
Cherenkov light will still be generated in scintillator, however in LAB+PPO it will be insignificant compared to the light emitted by the relaxing of vibrational states of the scintillating medium itself.

\subsection{Scintillation Light}

As a charged particle moves through a scintillator, electronic energy levels in the scintillator are excited which can subsequently relax by emitting a visible photon.
In the case of LAB+PPO in SNO+, this results in about 500 detected hits per MeV of deposited energy compared to the roughly 10 hits per MeV in light water.
The much larger number of photons results in much higher energy resolution c.f. a water detector, which is very advantageous to analyses that rely heavily on energy reconstruction.

Unlike Cherenkov light, scintillation light is emitted isotropically, since the relaxing of the vibrational state is uncorrelated with the direction of the incident particle.
In a similar vein, scintillators are optically less transparent than water (typically) leading to increased scatter (isotropy) in the emitted light.
With much of the Cherenkov light scattered by the scintillator, and the remaining Cherenkov photons burried under two orders of magnitude more scintillation photons, there is effectively no detectable Cherenkov light in the events.
This leads to the significant downside of being unable to reconstruct direction, and therefore being unable to disentangle events based on directionality. 
For solar neutrinos this is a significant loss, because typically there is no other class of events correlated with the direction of the sun.
This is applicable both in analyses where solar neutrinos are of interest and should be selected, and also where solar neutrinos would be considered a background and should be rejected.

The final significant difference between Cherenkov and scintillation light is the minimum energy threshold.
Cherenkov light relies on the charged particle being very relativisitic (exceeding the local speed of light), meaning that $\alpha$ particles are typically invisible, while betas must exceed 0.7 MeV, to emit any light.
Scintillation, on the other hand, has a much lower threshold such that typically the maximum rate of the detector electronics are the limiting factor for data acquisition, not the threshold of the medium.
In the case of scintillation, both alphas and betas are visible, and analysis thresholds of hundreds of keV (electron hypothesis) are not unreasonable. 
For solar neutrinos, these lower thresholds give access to virtually all types of neutrino fluxes, not just the $^8$B neutrinos accessible by water Cherenkov detectors.
The Borexino \cite{borexino} and KamLAND \cite{kamland} experiments have made excellent solar neutrino measurements with scintillator, and SNO+ plans similar analyses for scintillator phase.

\section{T\textsc{HEIA}}

T\textsc{HEIA} \cite{asdc_paper} is a proposed large scale (50-100kT) neutrino detector which aims to support a diverse platform of neutrino physics. 
Of particular relevance to this thesis is the goal of doing high precision measurements of solar neutrinos \cite{theia_solar}.
The detector design is still in the conceptual phase, but lessons learned from SNO+ and other liquid scintillator detectors are being incorporated, and the optical detection scheme remains roughly unchanged except for potential upgrades to technologies similar to PMTs.
The basic plan is two-fold: build a very large detector to maximize interaction rate and enhance fiducial volume shielding, and develop a novel target medium that combines the advantages of Cherenkov and scintillation light.
The former is primarily an engineering and/or funding challenge, while the latter presents an interesting research topic.

\subsection{Combined Scintillation and Cherenkov Signal}

There are multiple technologies in development with the goal of having simultanously detectable Cherenkov and scintillation signals, however for this thesis I will focus on one in particular: water-based liquid scintillators (WbLS).
WbLS mixes a traditional scintillator, LAB+PPO, an oil, with water at some configurable ratio that is in large part water.
In principle the fraction of scintillator results in a fraction of total light output from a pure scintillator, so low scintillator fractions will have a comparable number of scintillation photons to cherenkov photons. 
This leads to enhanced energy resolution compared to only Cherenkov light, while allowing for the possibility of observing Cherenkov geometry in the pattern of detected hits, and thereby obtaining direction information.
At higher scintillation fractions, the relatively fewer Cherenkov photons might be difficult to identify on top of the isotropic scintillation light.
In this case one may leverage the fact that Cherenkov light is very prompt (picoseconds) compared to scintillators (LAB+PPO has a short time constant of a few nanoseconds) to potentially identify Cherenkov photons as the most prompt hits.
Finally, being mostly water, WbLS should have longer scattering and attenuation lengths than a pure scintillator, allowing larger volumes to be built before either effect becomes a limiting factor.
This seems to identify WbLS as an ideal material for combining the long attentuation lengths of water with the directionality of Cherenkov light and higher energy resolutions of scintillating materials.
Further discussion of WbLS and a characterization of low scintillator fractions can be found in \Cref{ch:wbls}.


