\chapter{Cherenkov Source Bonding Procedure}
\label{chap:bondproc}

The detailed bonding procedures for the most critical step in Cherenkov source construction are described here. 
\begin{enumerate}
\item Clean up and clear the workspace.  
\item Assemble all equipment and make sure everything is there. Anything that will touch the Weld-On~\#40 or the blocks will need to be UHV cleaned. The following is needed:
\begin{itemize}
\item UHV cleaned cups, Teflon stirring stick, clamps (for source and reference piece), razor blades and long tweezers.
\item UHV cleaned flat screwdriver which can be used to pry the two block apart in case of a serious issue with the bond.
\item Machined and UHV cleaned acrylic blocks, with the decay chamber polished and the bond-plane sanded. All inside surfaces should have been masked and the masking has been verified to be completely dry. The positions of the clamps need to be marked on the top block.
\item UHV cleaned acrylic alignment pins in contrasting color.
\item UHV cleaned reference UVA acrylic blocks.
\item Cleaned precision scale.
\item Fresh Weld-On~\#40, components A and B. Make sure to inspect the color of the Weld-On~\#40, it should not be yellow. 
\item New syringe to add component B to A.
\item Vacuum pump system.
\item Valve system for compressed air. This includes a filter for the compressed air-line.
\item Flashlight to inspect bond-plane.
\item Reference blocks.
\item Camera in place and taking pictures.
\item Smaller camera for close-up shots.
\item Pen and clean-room paper for notes.
\item Weld-On~\#40 suction system. This will be used to suck the excess Weld-On~\#40 out of the decay chamber before it has time to cure. 
\end{itemize}
\item Unpack everything, position on table.
\item Position the two acrylic blocks in the correct position on the table. One should be lying on its back, with the alignment pins in place. The other block should be standing on its side in such a way that rotation will be minimal when joining the two blocks.
\item Inspect the bond-surface. Remove any clear debris with clean-room wipe. 
\item Inspect the inside edges of the bond-plane and remove any masking from the bond-plane using the edge of a razor. Any masking that remains would be locked inside the bond-plane. 
\item Carefully open the Weld-On~\#40 away from the two acrylic blocks. The opening  might release some spray droplets that could fall on the blocks. Be aware that a strong odor will be present once the Weld-On~\#40 is opened. 
\item Using a precision scale mix exactly 5 parts in 100 grams of the Weld-On~\#40 catalyst (component ``B'') with the Weld-On~\#40 glue (component ``A'') at room temperature. Component ``B'' should be added slowly and over the entire surface area of component ``A'' to prevent initial hardening of the glue. Stir very thoroughly with a Teflon stick until the mixture is completely mixed. We used 200 grams of component A and a more shallow mixing cup.
\item Immediately after stirring, place the mixture in a vacuum chamber to remove air bubbles from the mixture. The pressure should be reduced to at least 10 Torr. The air bubbles in the mixture rise to the top and begin the mixture will seem to froth. At this point the pressure should be released and the air bubbles will pop and begin to disappear. This process should be repeated at least three times, or until the air bubbles are almost completely gone.
\item Pour the glue on one piece.  A few cm thick layer, closer to the decay chamber edge than to the outside of the blocks. About 1~cm away from edge. Follow the line of the decay chamber edge. Do not worry about cement leaking over the edges. The blue masking will remove the cement left on the inside of the source and the outside of the source will be machined again. The more cement there is on the masking, the harder it is to remove it.
\item Bond the acrylic pieces. Two people lift the  top block together while the third person stands on the side to guide the correct positioning of the pins compared to the top block. Make sure that the blocks are oriented correctly. Fit the top piece on top of bottom piece using the guidance pins.
\item Quickly use the flashlight to inspect the glue plane. There are a few seconds where the blocks could still be pulled apart if a large contamination is spotted in time. Use the flat screw-driver to do so if needed.
\item Clamp the acrylic pieces together. Start by adding the middle clamp. Make sure that the clamp is oriented away from the neck-opening. Then, add the corner clamps. Their position is marked by red dots on the top block. Check with flashlight for bubbles while clamping is happening. Do not over-clamp.
\item Using a suction-system, suck out as much Weld-On~\#40 out of the decay chamber as possible. This will make the removing of the masking much easier
\item In order to ensure the glue dries,  blow compressed air into the decay chamber while it is clamped. This will push the vapor from the Weld-On~\#40 out of the decay chamber and allow the glue to dry properly. Do not let the compressed air line touch the blocks.
\item Bond the reference blocks and clamp with metal clamp.
\item Once the glue is dried (about 1 hour), remove the masking. Test the texture of the outside glue-blobs to make sure the glue has hardened. Using some sort of tweezers/grabbers, peel the masking off from the inside of the source. Twisting technique.
\item Let the glue cure for minimum 48 hours.
\item Inspect the source to make sure that all masking is gone. If needed, put it through the UHV shop before the next machine cycle.
\end{enumerate}
