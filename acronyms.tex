\chapter*{List of Acronyms}
\addcontentsline{toc}{chapter}{List of Acronyms}
\begin{labeling}{{\labppo}}
\item[ADC] Analog to digital converter --- samples an analog signal to produce an integer value corresponding to discrete intervals of a fixed range.
\item[AV] Acrylic vessel --- contains the target material for SNO and {\snop}.
\item[CAEN] An Italian company that produces high speed signal digitizers.
\item[CC] Charged current --- a type of weak interaction involving $W^{\pm}$ bosons.
\item[CHESS] Cherenkov / scintillation separation --- refers specifically to a benchtop experiment described in this thesis.
\item[CNO] Carbon, nitrogen, and oxygen (neutrinos) --- a fusion path within the Sun that has not yet been well measured, but produces a measurable flux of neutrinos.
\item[DAQ] Data acquisition --- a generic term for the electronics, data storage devices, and software that records data from a detector.
\item[ES] Elastic scatter --- a type of physical interaction where only the energy and momentum of particles change.
\item[FECD] Front-end card, digital --- allows signals from devices other than the standard PMT array to be connected to the DAQ of SNO and {\snop}.
\item[FWHM] Full width half max --- the width of a distribution at half its peak value.
\item[GEANT4] A collection of geometry modeling, particle tracking, and physical process simulation code used by many experiments.
\item[ITR] In-time ratio --- refers to a ratio of photon detection times within a prompt window to the entire event.
\item[LAB] Linear alkylbenzene --- a poor scintillator, yet easy to produce commercially, and can be augmented with fluorescent molecules to increase the light yield. 
\item[{\labppo}] Linear alkylbenzene with a primary flour 2,5-diphenyloxazole at 2~g/L --- a scintillating liquid with high light yield identified by {\snop} and used by other experiments.
\item[LBNL] Lawrence Berkeley National Lab --- the facilities where the Cherenkov source was constructed.
\item[LED] Light emitting diode --- emits photons of a particular energy as electrons are driven across a band gap in a semiconductor.
\item[LS] Liquid scintillator --- a generic term for a liquid that emits light as charged particles deposit energy.
\item[MC] Monte Carlo --- a technique of modeling a random process with an ensemble of randomly sampled trajectories.
\item[MSW] Mikheyev-Smirnov-Wolfenstein --- all contributed to the model of coherent forward scattering of neutrinos off of electrons that describes the observed survival probability of solar neutrinos.
\item[NAA] Neutron activation analysis --- irradiating a sample with a high neutron flux to convert isotopes with long halflives to less stable isotopes via neutron absorption. The decay of the less stable isotopes is easier to measure, and one can then infer the abundance of the parent isotopes. 
\item[NC] Neutral current --- a type of weak interaction involving $Z_0$ bosons.
\item[NCD] Neutral current detector --- devices designed and used by SNO to detect neutrons with high efficiency using $^3$He proportional counters.
\item[NHits] Number of hits, where a hit is defined as a photon detected by a PMT.
\item[PDF] Probability density function --- used in statistics to define probability with respect to a set of parameters.
\item[PE] Photoelectron --- an electron liberated from a metal via the photoelectric effect.
\item[PMNS] Pontecorvo-Maki-Nakagawa-Sakata --- all contributed to the understanding of the matrix that defines the change of basis from neutrino mass states to neutrino flavor states.
\item[PMT] Photomultiplier tube --- utilizes the photoelectric effect to convert photons into photoelectrons, which are amplified to detectable voltages by secondary emission after being accelerated into a series of metal plates called dynodes.
\item[PSUP] PMT support structure ---  a geodesic sphere that holds the PMT array for SNO and {\snop}.
\item[QE] Quantum efficiency --- the probability of a PMT photocathode emitting a photoelectron when a photon is absorbed.
\item[QHL] A measure of observed PMT charge (Q) with high gain (H) in a long (L) integration window used by the SNO and {\snop} detectors.
\item[RAT] The {\snop} Monte Carlo and analysis package implemented on top of GEANT4 and ROOT and used to model the {\snop} detector.
\item[RAT-PAC] The open source version of the RAT package, being used by many experiments, including \textsc{Theia}.
\item[ROI] Region of interest --- a subset of the total available parameter space that is considered for an analysis.
\item[ROOT] An analysis, plotting, and data storage library used by many experiments.
%\item[SNO] The Sudbury Neutrino Observatory --- used a heavy water target sensitive to all neutrino flavors to solve the solar neutrino problem.
%\item[{\snop}] The upgrade of the SNO detector that will search for neutrinoless double beta decay using a liquid scintillator target.
\item[SNOLAB] The facility that grew from the construction of a lab space underground for SNO that now contains {\snop} and other experiments.
\item[SNOMAN] The SNO Monte Carlo and analysis package written in FORTRAN and used to model the SNO detector.
\item[SNP] Solar neutrino problem --- Refers to the then-unexplained flux deficit observed in early studies of solar neutrinos.
\item[SPE] Single photoelectron --- PMT behavior is typically quantified in terms of response to single photoelectrons.
\item[SSM] Standard solar model --- a description of the entire Sun, including the fusion reactions that emit neutrinos.
\item[TPB] Tetraphenyl butadiene --- a wavelength shifter that absorbs ultra violet photons and reemits blue visible photons.
\item[TTS] Transit time spread --- A figure of merit for photodetectors that defines the uncertainty in photon hit time with respect to the generated voltage pulse.
\item[UHV] Ultra high vacuum --- UHV applications require exceptional cleanliness, and UHV cleaning facilities were utilized here.
\item[VME] Versa Module Europa (bus) --- an electrical and mechanical specification for connecting digital electronics.
\item[WbLS] Water-based liquid scintillator --- a scintillating material made from {\labppo} and water.
\item[WbLSdaq] The DAQ used by the CHESS experiment.
\end{labeling}
