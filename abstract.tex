\begin{abstract}

This thesis represents a comprehensive study of solar neutrinos spanning three generations of solar neutrino detection techniques.
Data from the Sudbury Neutrino Observatory (SNO), a heavy water detector instrumental in demonstrating flavor change in the neutrino sector, is used to set a limit on the lifetime of massive neutrino states within a model where said states can decay.
The analysis of SNO data finds a limit of $k_2>8.08 \times 10^{-5}$ s/eV for the lifetime of $\nu_2$.
Combining this limit with constraints from all other solar neutrino experiments, a new global best limit is found at $k_2>1.92 \times 10^{-3}$ s/eV for the lifetime of $\nu_2$.

The upgrade of the SNO detector, SNO+, is currently in the process of migrating to a liquid scintillator target material.
Prior to this, SNO+ produced a physics dataset with light water as the target, which was analyzed to extract the flux of $^8$B solar neutrinos.
This analysis resulted in a detected flux of $\Phi_{^8\mathrm{B}} = 6.17^{+1.04}_{-0.98}$(stat)$^{+0.21}_{-0.22}$(syst).
To enable the move from water to scintillator, the design and evaluation of a Cherenkov optical calibration source for SNO+ is presented.

With an eye on the future, the capabilites of the proposed \textsc{Theia} experiment are discussed, with particular attention to the novel water-based liquid scintillator target being evaluated for use by that experiment.
A small scale experiment to evaluate the capability to distinguish scintillation and Cherenkov light in this target is described. 
Results are presented demonstrating the potential for identification of Cherenkov photons by timing and intensity information for this, and for more traditional scintillators such as LAB+PPO.

\end{abstract}
