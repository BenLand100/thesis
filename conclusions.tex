\chapter{Conclusions}

Solar neutrino physics is now in the realm of precision measurements.
Advances in detector technology have taken us from the early days of simply counting neutrino interactions to modern real-time detection and vertex reconstruction able to infer incident neutrino energy and flavor on a statistical basis.
With the demonstration of flavor change in the neutrino sector and confirmation of the MSW solution to the solar neutrino problem, solar neutrinos can now be used as probes for new and interesting physics.
Because flavor change implies neutrinos have mass, there is a possibility that neutrinos could decay to less massive particles.
In \Cref{ch:lifetime} I used the current understanding of the $^8$B flux of solar neutrinos along with data from the SNO detector to set a limit on potential neutrino decay.
Combining this result from SNO with constraints from other solar neutrino experiments led to a new best limit on the lifetime of neutrino mass state $\nu_2$.

Moving forward, the primary goal of solar neutrino experiments has been increased sensitivity to lower energy solar neutrino fluxes.
Included in these fluxes are the CNO neutrinos, which represent a fusion path in the solar core that has not yet been measured directly.
To measure these fluxes, which are typically at energies below the Cherenkov threshold in water, a move away from water Cherenkov detectors to liquid scintillator detectors was necessitated.
The upgrade of SNO, {\snop}, is currently in the process of making such a change.

Scintillator has its own unique set of challenges when it comes to measuring solar neutrinos.
The light production mechanism is not as well understood in scintillators as Cherenkov light is in dialectic materials.
This requires novel calibration techniques to evaluate detector performance independent of the behavior of the target medium.
The Cherenkov optical calibration source described in \Cref{ch:chsrc} uses tagged decays of \Li to generate Cherenkov light in acrylic.
This light is independent of the properties of the target material in which the source was deployed.
This Cherenkov light source can be used to understand detector performance in a way that is decoupled from the optics of the target material, and successful calibration of this source for use in {\snop} has been shown in \Cref{ch:chsrc}.

The directional correlation of electrons elastically scattered by neutrinos was successfully used to extract the solar neutrino flux from {\snop} water phase, as presented in \Cref{ch:es}.
This directional information was available because Cherenkov light is emitted at an angle with respect to the direction of the energetic particle creating it.
In that analysis, the fact that no background sources are correlated with the direction of the Sun was used to detect the electrons elastically scattered by $^8$B solar neutrinos down to $5$ MeV of visible energy.
The inferred flux of solar neutrinos was consistent with measurements from other experiments, and an extraction of the spectrum of elastically scattered electrons indicated SNO+ has achieved very low backgrounds down to $6$ MeV of visible energy.

Unlike Cherenkov light, scintillation light production is uncorrelated with the direction of the energetic particle.
This means that analyses in scintillator detectors cannot leverage the directional correlation of electrons elastically scattered by solar neutrinos.
This lack of directional information in scintillator is a significant limiting factor to future solar neutrino experiments.
Some mechanism for simultaneously achieving the low energy thresholds of scintillator and the directional information of Cherenkov light would be ideal.
The \textsc{Theia} experiment proposes to do just that by using a novel target material that allows for simultaneous identification of scintillation and Cherenkov photons.
Water-based liquid scintillator (WbLS) has been identified as a possible candidate target material, as its scintillation photon yield can be tuned.
An evaluation of this material, along with other more traditional liquid scintillators, with the CHESS detector in \Cref{ch:wbls} indicates a real possibility for simultaneous detection of Cherenkov and scintillation light using either time-based or intensity-based photon discrimination.
Combining both time and intensity information should result in optimal Cherenkov photon discrimination, and should be explored in future studies.
If the simultaneous detection of Cherenkov and scintillation light proves possible at large scale, this opens the door to unprecedented sensitivity to the lower energy solar neutrino fluxes and the rich probe of physics they represent.
