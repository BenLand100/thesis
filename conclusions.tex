\chapter{Conclusions}

Solar neutrino physics is now in the realm of precision measurements.
Advances in detector technology have taken us from the early days of simply counting neutrino interactions to modern real-time detection and vertex reconstruction able to infer incident neutrino energy and flavor on a statistical basis.
With the demonstration of flavor change in the neutrino sector and confirmation of the MSW solution to the solar neutrino problem, solar neutrinos can now be used as probes for new and interesting physics.
Because flavor change implies neutrinos have mass, there is a possibility that neutrinos could decay to less massive particles.
In \Cref{ch:lifetime} I used the current understanding of the $^8$B flux of solar neutrinos along with data from the SNO detector to set a limit on potential neutrino decay.
Combining this result from SNO with constraints from other solar neutrino experiments led to a new best limit on the lifetime of neutrino mass state $\nu_2$.

Moving forward, the primary goal of solar neutrino experiments has been increased sensitivity to lower energy solar neutrino fluxes.
Included in these fluxes are the CNO neutrinos, which represent a fusion path in the solar core that has not yet been measured directly.
To measure these fluxes, which are typically at energies below the Cherenkov threshold in water, a move away from water Chernekov detectors to liquid scintillator detectors was necessitated.
The upgrade of SNO, {\snop}, is currently in the process of making such a change.

Scintillator has its own unique set of challenges when it comes to measuring solar neutrinos.
The light production mechanism is not as well understood in scintillators as Cherenkov light is in water.
This requires novel calibration techniques to evaluate detector performance independent of the behavior of the target medium, such as the Cherenkov source optical calibration device described in \Cref{ch:chsrc}.
Also, scintillation light does not carry the directional information Cherenkov light does.
This means that analyses in scintillator detectors cannot leverage the directional correlation between electrons elastically scattered by solar neutrinos that was so effective in extracting the solar neutrino flux from {\snop} water phase, as shown in \Cref{ch:es}.

This lack of directional information in scintillator is a significant limiting factor to future solar neutrino experiments.
Some mechanism for simultaneously achieving the low energy thresholds of scintillator and the directional information of Cherenkov light would be ideal.
The \textsc{Theia} experiment proposes to do just that by using a novel target material: water-based liquid scintillator (WbLS).
An evaluation of this material, along with other more traditional liquid scintillators, in \Cref{ch:wbls} indicates a real possibility for simultaneous detection of Cherenkov and scintillation light using either time based or intensity based photon discrimination.
If the simultaneous detection of Cherenkov and scintillation light proves possible at large scale, this opens the door to unprecedented sensitivity to the lower energy solar neutrino fluxes and the rich probe of physics they represent.
